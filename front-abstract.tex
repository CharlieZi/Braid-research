
\begin{abstract}
	We conducted an experiment and put forward several modifications, based on the method proposed in Hachemi’s research of metal wire braid reinforced hose, in order to enhance the constitutive theory, which is discovered not keeping up with the experimental results when it shows far more non-lineal mechanical behavior than the theory predict. We introduced a “modifying matrix”, from composites mechanics, to detach inter-wire contacts from wire elongation, seldom considered before. We also proposed a hypothesis opposite to Hachemi’s: the hose’s braid angle decreased linearly, applied displacement load with constant loading rate, rather than locked at a certain degree. So that we introduce a modification coefficient \textit{k,} accelerating the decrease of braid angle to match the linearity in force-displacement curve. Lateral contact is considered to be the factor of excessively decreased braid angle when the calculated curve perfectly meet the experimental one, with suitable \textit{k}.

\end{abstract}




